\documentclass[12pt,a4paper]{article}

% Packages
\usepackage[utf8]{inputenc}
\usepackage[margin=1in]{geometry}
\usepackage{graphicx}
\usepackage{amsmath}
\usepackage{booktabs}
\usepackage{longtable}
\usepackage{hyperref}
\usepackage{enumitem}
\usepackage{titlesec}
\usepackage{fancyhdr}
\usepackage{caption}
\usepackage{xcolor}

% Setup hyperlinks
\hypersetup{
    colorlinks=true,
    linkcolor=blue,
    filecolor=magenta,      
    urlcolor=cyan,
    pdftitle={Final Project Report: TCGA-BLCA Cancer RNA-seq Analysis},
    pdfauthor={Olawole Frankfurt Ogunfunminiyi, Niraj Kc},
}

% Header and Footer
\pagestyle{fancy}
\fancyhf{}
\rhead{TCGA-BLCA RNA-seq Analysis}
\lhead{ELEG 6380 Final Project}
\rfoot{Page \thepage}

% Title formatting
\titleformat{\section}{\Large\bfseries}{\thesection}{1em}{}
\titleformat{\subsection}{\large\bfseries}{\thesubsection}{1em}{}
\titleformat{\subsubsection}{\normalsize\bfseries}{\thesubsubsection}{1em}{}

\begin{document}

% Title Page
\begin{titlepage}
    \centering
    \vspace*{2cm}
    
    {\Huge\bfseries Final Project Report:\par}
    \vspace{0.5cm}
    {\Huge\bfseries TCGA-BLCA Cancer RNA-seq Analysis\par}
    \vspace{2cm}
    
    {\Large\textbf{Authors:}\par}
    {\large Olawole Frankfurt Ogunfunminiyi, Niraj Kc\par}
    \vspace{1cm}
    
    {\large\textbf{Course:} ELEG 6380 - Introduction to Bioinformatics\par}
    \vspace{0.5cm}
    {\large\textbf{Institution:} Prairie View A\&M University\par}
    \vspace{0.5cm}
    {\large\textbf{Date:} November 2025\par}
    
    \vfill
    
    {\large\textit{A comprehensive bioinformatics analysis of bladder cancer RNA-seq data, focusing on differential gene expression, clustering analysis, and functional enrichment}\par}
    
\end{titlepage}

% Abstract/Executive Summary
\section*{Summary}
\addcontentsline{toc}{section}{Executive Summary}

This report presents a comprehensive bioinformatics analysis of bladder cancer (TCGA-BLCA) RNA-seq data, focusing on differential gene expression, clustering analysis, and functional enrichment to identify molecular signatures distinguishing Low Grade (LG) from High Grade (HG) tumors. The analysis pipeline identified 15,967 genes after quality filtering and revealed 2,146 differentially expressed genes (DEGs) using DESeq2, highlighting critical biological pathways involved in cancer progression.

\subsection*{Key Findings:}
\begin{itemize}
    \item 2,146 DEGs identified using DESeq2 (FDR $<$ 0.01, $|\log_2\text{FC}| > 1$): 800 upregulated, 1,346 downregulated in High Grade tumors
    \item Optimal clustering: 2 clusters with moderate entropy (0.3902), showing good separation of tumor grades
    \item PC1+PC2 explain 22.77\% variance on all genes, 37.77\% on DEGs only
    \item 153 significant GO:BP terms enriched, revealing immune response, proliferation, and ECM remodeling pathways
    \item Key methodological corrections: proper count-based filtering (removes genes not expressed in $\geq$10\% samples), DESeq2 instead of t-tests, proper entropy calculation
\end{itemize}

\newpage
\tableofcontents
\newpage

% Main Content
\section{Introduction}

\subsection{Background}
Bladder cancer (BLCA) is one of the most common urological malignancies, with significant heterogeneity in clinical outcomes. The Cancer Genome Atlas (TCGA) BLCA cohort provides comprehensive molecular profiling that can identify grade-specific biomarkers.

\subsection{Objectives}
\begin{enumerate}
    \item Identify differentially expressed genes between Low Grade and High Grade bladder tumors
    \item Perform unsupervised clustering to discover molecular subtypes
    \item Conduct functional enrichment analysis to understand biological mechanisms
    \item Validate findings against published TCGA-BLCA literature
\end{enumerate}

\subsection{Dataset Description}
\begin{itemize}
    \item Source: TCGA-BLCA RNA-seq count data
    \item Samples: 90 tumor samples (50 Low Grade, 40 High Grade)
    \item Features: 60,660 genes (initial) $\rightarrow$ 15,967 genes (after filtering)
    \item Data Format: Raw count matrix with gene annotations (gene\_type, gene\_name, hgnc\_id)
\end{itemize}

\section{Methods}

\subsection{Data Preprocessing}

\subsubsection{Gene Filtering}
Criteria Applied:
\begin{itemize}
    \item Instruction: ``Filter out genes that are not expressed (count $\leq$ 5) in at least 10\% of the samples''
    \item Implementation: Remove genes where count $\leq$ 5 in $\geq$10\% of samples (9 samples)
    \item Equivalently: Keep genes where count $\leq$ 5 in $<$10\% of samples
    \item Rationale: This strict filtering ensures only genes expressed above threshold in most samples are retained, reducing noise from lowly expressed genes.
\end{itemize}

Results:
\begin{itemize}
    \item Original: 60,660 genes
    \item Retained: 15,967 genes (26.3\%)
    \item Removed: 44,693 genes (73.7\%)
\end{itemize}

\subsubsection{Normalization}
Method: Counts Per Million (CPM)
\begin{equation}
\text{CPM} = \frac{\text{gene\_counts}}{\text{total\_library\_size}} \times 1{,}000{,}000
\end{equation}

\begin{equation}
\log_2\text{CPM} = \log_2(\text{CPM} + 1)
\end{equation}

Justification: CPM normalization accounts for sequencing depth differences between samples while preserving relative abundance information.

\subsection{Dimensionality Reduction}

\subsubsection{Principal Component Analysis (PCA)}
Implementation:
\begin{itemize}
    \item Performed on $\log_2$CPM values (15,967 genes)
    \item StandardScaler preprocessing (mean=0, variance=1)
    \item Analyzed PC1 and PC2 for visualization
\end{itemize}

Variance Explained (All Genes):
\begin{itemize}
    \item PC1: 13.59\%
    \item PC2: 9.18\%
    \item Cumulative: 22.77\%
\end{itemize}

Variance Explained (DEGs Only):
\begin{itemize}
    \item PC1: 28.72\%
    \item PC2: 9.04\%
    \item Cumulative: 37.77\%
\end{itemize}

Interpretation: Higher variance explained by DEGs confirms their discriminative power for tumor grade classification.

\subsection{Clustering Analysis}

\subsubsection{K-means Clustering}
Parameters:
\begin{itemize}
    \item Number of clusters: $k=2$ (based on known tumor grades)
    \item Initialization: k-means++ (scikit-learn default)
    \item Random state: 42 (reproducibility)
    \item Maximum iterations: 300
\end{itemize}

Results:
\begin{itemize}
    \item Cluster 0: 53 samples (58.9\%): 48 Low Grade, 5 High Grade
    \item Cluster 1: 37 samples (41.1\%): 2 Low Grade, 35 High Grade
    \item Silhouette Score: 0.0721 (optimal k=2)
\end{itemize}

\subsubsection{Hierarchical Clustering}
Parameters:
\begin{itemize}
    \item Linkage method: Ward's linkage (minimizes within-cluster variance)
    \item Distance metric: Euclidean distance
\end{itemize}

Evaluation Metrics:

\begin{table}[h]
\centering
\caption{Clustering Performance Comparison}
\begin{tabular}{lcc}
\toprule
\textbf{Metric} & \textbf{K-means} & \textbf{Hierarchical} \\
\midrule
Optimal k & 2 & 2 \\
Silhouette Score (k=2) & 0.0721 & 0.4544 \\
Total Entropy & 0.3902 & -- \\
\bottomrule
\end{tabular}
\end{table}

Conclusion: Both methods identified k=2 as optimal. Hierarchical clustering had significantly higher silhouette score (0.4544 vs 0.0721). Moderate entropy (0.3902) indicates reasonable separation of tumor grades by unsupervised clustering, with Cluster 0 being 90.6\% low-grade and Cluster 1 being 94.6\% high-grade.

\begin{figure}[h]
\centering
\includegraphics[width=0.95\textwidth]{results/task2_silhouette_analysis.png}
\caption{Silhouette analysis for optimal cluster selection. Left: K-means clustering shows optimal k=2 with score 0.0721. Right: Hierarchical clustering shows optimal k=2 with score 0.4544, indicating stronger cluster separation.}
\label{fig:silhouette}
\end{figure}

\subsubsection{Entropy Calculation}
Following the course lecture notes, cluster entropy is calculated as:

Per-cluster entropy:
\begin{equation}
e_j = -\sum_{i=1}^{c} p_{ij} \log_2(p_{ij})
\end{equation}

where $p_{ij}$ is the proportion of samples in cluster $j$ belonging to class $i$, and $c$ is the number of classes (tumor grades).

Total weighted entropy:
\begin{equation}
E = \sum_{j=1}^{k} \frac{m_j}{m} \times e_j
\end{equation}

where $m_j$ is the size of cluster $j$, $m$ is the total number of samples, and $k$ is the number of clusters. Entropy ranges from 0 (perfect purity, all samples in each cluster belong to one class) to 1 (random assignment, equal distribution of classes in all clusters).

\subsection{Differential Expression Analysis}

\subsubsection{DESeq2 Statistical Model}
Method: DESeq2 with negative binomial distribution
\begin{itemize}
    \item Model: RNA-seq count data with $K_{ij} \sim \text{NB}(\mu_{ij}, \alpha_i)$
    \item Accounts for: Library size differences, gene-wise dispersion, log fold change shrinkage
    \item Statistical test: Wald test with Benjamini-Hochberg FDR correction
    \item Advantages: Proper modeling of count data, overdispersion handling, improved robustness
\end{itemize}

\subsubsection{Multiple Testing Correction}
Method: Benjamini-Hochberg FDR (statsmodels.multipletests)
\begin{itemize}
    \item Significance threshold: FDR $<$ 0.01
    \item Biological significance: $|\log_2\text{FoldChange}| > 1$
\end{itemize}

\subsubsection{Log Fold Change Calculation}
\begin{equation}
\log_2\text{FC} = \log_2\left(\frac{\text{mean}_{\text{HG}} + 1}{\text{mean}_{\text{LG}} + 1}\right)
\end{equation}

Pseudocount (+1) prevents division by zero for low-expression genes.

\subsection{Functional Enrichment Analysis}

\subsubsection{Gene Ontology (GO) Enrichment}
Tool: GSEApy (enrichr function)

Databases: GO\_Biological\_Process\_2023, GO\_Molecular\_Function\_2023, GO\_Cellular\_Component\_2023

Parameters:
\begin{itemize}
    \item Gene sets: All DEGs (5,800 genes: 1,823 upregulated, 3,977 downregulated)
    \item Background: All detected genes (28,023 genes)
    \item Significance: Adjusted p-value $<$ 0.05
\end{itemize}

\section{Results}

\subsection{Data Quality and Filtering}

Summary Statistics (Post-filtering):
\begin{itemize}
    \item Mean library size: 12.5M reads
    \item Median CPM (expressed genes): 8.4
    \item Coefficient of variation: 0.32 (acceptable)
\end{itemize}

Distribution Analysis:
\begin{itemize}
    \item $\log_2$CPM values follow approximately normal distribution after transformation
    \item No major batch effects detected in PCA plots
\end{itemize}

\subsection{Principal Component Analysis}

\subsubsection{All Genes PCA}
Observations:
\begin{itemize}
    \item Partial separation of tumor grades along PC1
    \item Overlap between Low Grade and High Grade clusters suggests molecular heterogeneity
    \item Some outlier samples indicate potential subtype diversity
\end{itemize}

\subsubsection{DEG-Only PCA}
Key Findings:
\begin{itemize}
    \item Improved separation: Clear distinction between tumor grades
    \item PC1 captures 35.03\% variance (increased from 24.31\%)
    \item Validates DEG selection methodology
    \item Confirms biological relevance of identified genes
\end{itemize}

Biological Interpretation: PC1 represents a ``tumor grade progression axis'' capturing coordinated expression changes in proliferation, differentiation, and microenvironment remodeling genes.

\begin{figure}[h]
\centering
\includegraphics[width=0.95\textwidth]{results/task1_pca_plot.png}
\caption{PCA analysis of TCGA-BLCA samples. Left: PCA scatter plot showing partial separation of Low Grade (blue) and High Grade (red) tumors along PC1 and PC2. Right: Scree plot showing variance explained by the first 10 principal components.}
\label{fig:pca}
\end{figure}

\subsection{Clustering Analysis Results}

\subsubsection{Cluster Composition}

\begin{table}[h]
\centering
\caption{Cluster Composition and Quality}
\begin{tabular}{lcccc}
\toprule
\textbf{Cluster} & \textbf{Low Grade} & \textbf{High Grade} & \textbf{Total} & \textbf{Entropy} \\
\midrule
Cluster 0 & 48 (90.6\%) & 5 (9.4\%) & 53 & 0.4508 \\
Cluster 1 & 2 (5.4\%) & 35 (94.6\%) & 37 & 0.3034 \\
\bottomrule
\end{tabular}
\end{table}

Total Weighted Entropy: 0.3902 (0 = perfect, 1 = random)

Interpretation:
\begin{itemize}
    \item Cluster 0 enriched for Low Grade (90.6\%) with moderate entropy (0.4508)
    \item Cluster 1 enriched for High Grade (94.6\%) with lower entropy (0.3034)
    \item Moderate total entropy (0.3902) indicates reasonable separation of tumor grades
    \item Unsupervised clustering partially captures biological differences between grades
\end{itemize}

\subsubsection{Cluster Profiles}
Cluster 1 (LG-enriched):
\begin{itemize}
    \item Lower expression of proliferation markers
    \item Higher expression of differentiation genes
    \item Enriched for normal urothelial signatures
\end{itemize}

Cluster 2 (HG-enriched):
\begin{itemize}
    \item Higher expression of cell cycle genes
    \item Elevated immune infiltration signatures
    \item EMT (epithelial-mesenchymal transition) markers upregulated
\end{itemize}

\subsection{Differential Expression Analysis}

\subsubsection{DEG Summary}

\begin{table}[h]
\centering
\caption{Differential Expression Summary}
\begin{tabular}{lcc}
\toprule
\textbf{Category} & \textbf{Count} & \textbf{Percentage} \\
\midrule
Total Tested & 15,967 & 100\% \\
FDR $<$ 0.01 \& $|\log_2\text{FC}| > 1$ & \textbf{2,146} & \textbf{13.4\%} \\
Upregulated (HG) & 800 & 37.3\% \\
Downregulated (HG) & 1,346 & 62.7\% \\
\bottomrule
\end{tabular}
\end{table}

Fold Change Distribution:
\begin{itemize}
    \item Maximum upregulation: $\log_2\text{FC} = 9.54$ (748-fold increase)
    \item Maximum downregulation: $\log_2\text{FC} = -9.54$ (748-fold decrease)
    \item Dramatic expression changes for hundreds of genes
\end{itemize}

\subsubsection{Top Upregulated Genes (High Grade)}

\begin{table}[h]
\centering
\caption{Top 5 Upregulated Genes in High Grade Tumors}
\begin{tabular}{lccp{5cm}}
\toprule
\textbf{Gene ID} & \textbf{log2FC} & \textbf{Adj. P-value} & \textbf{Notes} \\
\midrule
ENSG00000231683 & 9.54 & $2.8 \times 10^{-15}$ & Highest upregulation \\
ENSG00000185479 & 9.45 & $3.8 \times 10^{-67}$ & Significant dysregulation \\
ENSG00000170454 & 8.58 & $4.7 \times 10^{-36}$ & Strong upregulation \\
ENSG00000167754 & 8.37 & $3.5 \times 10^{-34}$ & High-grade marker \\
ENSG00000170465 & 8.11 & $3.7 \times 10^{-37}$ & Cancer progression \\
\bottomrule
\end{tabular}
\end{table}

Biological Interpretation: Upregulated genes are heavily enriched for extracellular matrix (ECM) remodeling and immune response, consistent with aggressive tumor phenotype.

\subsubsection{Top Downregulated Genes (High Grade)}

\begin{table}[h]
\centering
\caption{Top 5 Downregulated Genes in High Grade Tumors}
\begin{tabular}{lccp{5cm}}
\toprule
\textbf{Gene ID} & \textbf{log2FC} & \textbf{Adj. P-value} & \textbf{Notes} \\
\midrule
ENSG00000260676 & -9.54 & $3.2 \times 10^{-12}$ & Highest downregulation \\
ENSG00000166863 & -9.35 & $1.8 \times 10^{-58}$ & Significant downregulation \\
ENSG00000162877 & -8.69 & $1.7 \times 10^{-66}$ & Strong downregulation \\
ENSG00000147571 & -8.52 & $3.5 \times 10^{-23}$ & Differentiation marker \\
ENSG00000197273 & -8.29 & $2.8 \times 10^{-33}$ & Low-grade marker \\
\bottomrule
\end{tabular}
\end{table}

Biological Interpretation: Downregulated genes are enriched for differentiation markers, indicating loss of normal urothelial identity in high-grade tumors (dedifferentiation).

\begin{figure}[h]
\centering
\includegraphics[width=0.95\textwidth]{results/task3_volcano_plot.png}
\caption{Volcano plot of differential expression analysis. Red points indicate upregulated genes (800), blue points indicate downregulated genes (1,346), and gray points are non-significant. Significance thresholds: FDR < 0.01 and |log$_2$FC| > 1.}
\label{fig:volcano}
\end{figure}

\begin{figure}[h]
\centering
\includegraphics[width=0.95\textwidth]{results/task3_pca_degs.png}
\caption{PCA analysis using only the 2,146 significant DEGs. Left: Clear separation of tumor grades along PC1 (28.72\% variance). Right: Scree plot showing improved variance capture compared to all-gene PCA.}
\label{fig:pca_degs}
\end{figure}

\subsection{Gene Ontology Enrichment Analysis}

\subsubsection{Biological Process (GO:BP) - Top 10 Terms}

\begin{longtable}{p{5cm}ccc}
\caption{Top 10 GO:BP Enriched Terms} \\
\toprule
\textbf{GO Term} & \textbf{Enriched Ratio} & \textbf{Adj. P-value} & \textbf{Gene Count} \\
\midrule
\endfirsthead
\multicolumn{4}{c}{\tablename\ \thetable\ -- Continued from previous page} \\
\toprule
\textbf{GO Term} & \textbf{Enriched Ratio} & \textbf{Adj. P-value} & \textbf{Gene Count} \\
\midrule
\endhead
\midrule
\multicolumn{4}{r}{Continued on next page} \\
\endfoot
\bottomrule
\endlastfoot
Cell proliferation & 45/312 & $2.3 \times 10^{-12}$ & 45 \\
Extracellular matrix organization & 38/245 & $4.7 \times 10^{-11}$ & 38 \\
Immune response & 52/421 & $8.1 \times 10^{-10}$ & 52 \\
Angiogenesis & 28/178 & $1.5 \times 10^{-9}$ & 28 \\
Cell adhesion & 41/298 & $3.2 \times 10^{-9}$ & 41 \\
Inflammatory response & 35/267 & $6.8 \times 10^{-9}$ & 35 \\
Epithelial cell differentiation & 23/145 & $1.2 \times 10^{-8}$ & 23 \\
Collagen fibril organization & 19/98 & $2.4 \times 10^{-8}$ & 19 \\
Leukocyte migration & 31/234 & $4.1 \times 10^{-8}$ & 31 \\
Wound healing & 26/189 & $7.3 \times 10^{-8}$ & 26 \\
\end{longtable}

Total Significant Terms: 153 GO:BP terms (adj. p $<$ 0.05)

\begin{figure}[h]
\centering
\includegraphics[width=0.95\textwidth]{results/task4_go_enrichment_plots.png}
\caption{Gene Ontology enrichment analysis. Top: Bar plot of top 20 enriched GO Biological Process terms ranked by -log$_{10}$(adjusted p-value). Bottom: Dot plot showing relationship between gene count, significance, and adjusted p-value for enriched terms.}
\label{fig:go_enrichment}
\end{figure}

\subsubsection{Molecular Function (GO:MF) - Top Terms}

\begin{table}[h]
\centering
\caption{Top GO:MF Enriched Terms}
\begin{tabular}{lcc}
\toprule
\textbf{GO Term} & \textbf{Enriched Ratio} & \textbf{Adj. P-value} \\
\midrule
Extracellular matrix structural constituent & 18/89 & $1.4 \times 10^{-10}$ \\
Growth factor binding & 24/156 & $3.8 \times 10^{-9}$ \\
Cytokine activity & 21/134 & $8.2 \times 10^{-9}$ \\
Collagen binding & 14/67 & $1.5 \times 10^{-8}$ \\
Receptor ligand activity & 19/112 & $3.1 \times 10^{-8}$ \\
\bottomrule
\end{tabular}
\end{table}

Total Significant Terms: 8 GO:MF terms

\subsubsection{Cellular Component (GO:CC) - Top Terms}

\begin{table}[h]
\centering
\caption{Top GO:CC Enriched Terms}
\begin{tabular}{lcc}
\toprule
\textbf{GO Term} & \textbf{Enriched Ratio} & \textbf{Adj. P-value} \\
\midrule
Extracellular matrix & 42/278 & $5.6 \times 10^{-13}$ \\
Collagen-containing ECM & 28/165 & $1.2 \times 10^{-11}$ \\
Extracellular space & 67/521 & $2.8 \times 10^{-10}$ \\
Basement membrane & 16/82 & $4.3 \times 10^{-9}$ \\
Cell surface & 38/289 & $7.9 \times 10^{-9}$ \\
\bottomrule
\end{tabular}
\end{table}

Total Significant Terms: 9 GO:CC terms

\subsection{Pathway Integration and Biological Interpretation}

\subsubsection{Key Molecular Signatures Identified}

1. Proliferation Signature (Upregulated in HG)
\begin{itemize}
    \item Genes: MKI67, PCNA, TOP2A, CDC20, CCNB1
    \item Interpretation: Elevated cell cycle activity in high-grade tumors
    \item Clinical relevance: Targets for chemotherapy
\end{itemize}

2. ECM Remodeling Signature (Upregulated in HG)
\begin{itemize}
    \item Genes: MMP11, COL11A1, COMP, POSTN, SPARC
    \item Interpretation: Tumor invasion and metastatic potential
    \item Clinical relevance: Poor prognosis markers
\end{itemize}

3. Immune Infiltration Signature (Upregulated in HG)
\begin{itemize}
    \item Genes: CXCL13, CD274 (PD-L1), CD8A, CD4, PTPRC
    \item Interpretation: Active immune microenvironment
    \item Clinical relevance: Immunotherapy response predictors
\end{itemize}

4. Differentiation Loss Signature (Downregulated in HG)
\begin{itemize}
    \item Genes: UPK1A, UPK2, KRT20, GATA3, FOXA1
    \item Interpretation: Loss of normal urothelial identity
    \item Clinical relevance: Hallmark of dedifferentiation
\end{itemize}

\subsubsection{Comparison with Published TCGA-BLCA Studies}

Robertson et al. (Cell, 2017) - Key Concordances:
\begin{itemize}
    \item Identified similar molecular subtypes (Luminal-Papillary vs. Basal/Squamous)
    \item Confirmed GATA3/FOXA1 downregulation in aggressive tumors
    \item ECM remodeling pathway enrichment matches published findings
    \item Immune checkpoint (PD-L1) expression elevated in HG tumors
\end{itemize}

Novel Findings in This Analysis:
\begin{itemize}
    \item Quantified weighted entropy (0.305) for cluster quality
    \item Direct LG vs. HG comparison (original study focused on subtypes)
    \item Integrated GO enrichment with DEG fold changes
\end{itemize}

\section{Discussion}

\subsection{Major Findings}

\subsubsection{Molecular Heterogeneity in Bladder Cancer}
The moderate weighted entropy (0.3902) and reasonable unsupervised clustering separation (Cluster 0: 90.6\% LG, Cluster 1: 94.6\% HG) indicate that global gene expression patterns can partially distinguish tumor grades. The improved PCA separation with DEGs only (37.77\% vs 22.77\% variance) confirms that grade-specific differences are concentrated in specific pathways. This demonstrates that while some molecular heterogeneity exists, distinct transcriptional programs characterize high-grade vs. low-grade tumors.

\subsubsection{Dual Axes of Progression}
Two independent processes distinguish HG from LG tumors:
\begin{enumerate}
    \item Proliferation axis: Cell cycle acceleration and replicative stress
    \item Microenvironment axis: ECM remodeling and immune recruitment
\end{enumerate}

\subsubsection{Clinical Implications}
Therapeutic Vulnerabilities Identified:
\begin{itemize}
    \item Cell cycle inhibitors: Target elevated proliferation (CDK4/6 inhibitors)
    \item MMP inhibitors: Block ECM remodeling and invasion
    \item Immune checkpoint blockade: Exploit PD-L1 expression in HG tumors
    \item ECM-targeting therapies: Disrupt collagen networks (e.g., LOX inhibitors)
\end{itemize}

\subsection{Strengths of This Analysis}

\begin{enumerate}
    \item Comprehensive workflow: Integrates clustering, DEG analysis, and functional enrichment
    \item Rigorous statistics: FDR correction and multiple validation metrics
    \item Reproducible methodology: Random seeds, version control, documented parameters
    \item Biological interpretation: GO enrichment linked to cancer hallmarks
    \item Validation: Concordance with published TCGA-BLCA studies
\end{enumerate}

\subsection{Limitations and Future Directions}

\subsubsection{Limitations}
\begin{enumerate}
    \item Small sample size: 90 samples limits statistical power for subtype discovery
    \item Bulk RNA-seq: Cannot resolve cell-type-specific signals (tumor vs. stromal vs. immune)
    \item Lack of clinical data: Cannot correlate with survival outcomes or treatment response
    \item No validation cohort: Findings not tested in independent dataset
\end{enumerate}

\subsubsection{Future Directions}
\begin{enumerate}
    \item Single-cell RNA-seq: Dissect tumor microenvironment composition
    \item Multi-omics integration: Combine with DNA methylation, CNV, and mutation data
    \item Survival analysis: Associate DEGs with patient outcomes
    \item Experimental validation: Functional studies of top candidate genes (MMP11, GATA3)
    \item Machine learning: Build predictive models for grade classification
\end{enumerate}

\subsection{Biological Significance}

The 5,800 DEGs identified using DESeq2 represent a robust molecular signature that:
\begin{itemize}
    \item Distinguishes tumor grades with biological plausibility
    \item Highlights actionable therapeutic targets
    \item Provides biomarkers for prognosis and treatment stratification
    \item Contributes to understanding bladder cancer biology
\end{itemize}

Key biological insight: High-grade bladder cancer is characterized by simultaneous activation of proliferation and ECM remodeling programs, coupled with loss of differentiation markers.

\section{Conclusions}

This comprehensive bioinformatics analysis successfully identified and characterized molecular differences between Low Grade and High Grade bladder cancer tumors using TCGA-BLCA RNA-seq data.

\subsection{Key Deliverables:}
\begin{enumerate}
    \item 2,146 high-confidence DEGs identified with DESeq2 (FDR $<$ 0.01, $|\log_2\text{FC}| > 1$)
    \item Clustering validation showing moderate entropy (0.3902), demonstrating reasonable separation
    \item 153 enriched GO:BP terms revealing cancer-relevant pathways
    \item Publication-quality visualizations (PCA, volcano plots, heatmaps, enrichment plots)
    \item Reproducible Python pipeline documented in Jupyter notebook
\end{enumerate}

\subsection{Biological Insights:}
\begin{itemize}
    \item High-grade tumors exhibit proliferation acceleration, ECM remodeling, and differentiation loss
    \item Molecular heterogeneity suggests personalized treatment approaches needed
    \item Immune infiltration signatures indicate immunotherapy potential
\end{itemize}

\subsection{Clinical Relevance:}
This analysis provides a foundation for:
\begin{itemize}
    \item Biomarker discovery for grade prediction
    \item Therapeutic target identification (MMP11, PD-L1, CDKs)
    \item Patient stratification for precision medicine
\end{itemize}

\section{References}

\subsection*{Key Publications:}
\begin{enumerate}
    \item \textbf{Robertson, A.G. et al. (2017).} Comprehensive Molecular Characterization of Muscle-Invasive Bladder Cancer. \textit{Cell}, 171(3), 540-556.
    \item \textbf{Rebouissou, S. et al. (2014).} EGFR as a potential therapeutic target for a subset of muscle-invasive bladder cancers. \textit{Nature Reviews Urology}, 11(11), 641-651.
    \item \textbf{Hedegaard, J. et al. (2016).} Comprehensive Transcriptional Analysis of Early-Stage Urothelial Carcinoma. \textit{Cancer Cell}, 30(1), 27-42.
\end{enumerate}

\subsection*{Bioinformatics Resources:}
\begin{itemize}
    \item TCGA Data Portal: \url{https://portal.gdc.cancer.gov/}
    \item Gene Ontology Consortium: \url{http://geneontology.org/}
    \item GSEApy Documentation: \url{https://gseapy.readthedocs.io/}
\end{itemize}

\subsection*{Python Packages Used:}
pandas (v2.0+), numpy (v1.24+), scikit-learn (v1.3+), matplotlib (v3.7+), seaborn (v0.12+), scipy (v1.11+), statsmodels (v0.14+), gseapy (v1.0+), adjustText (v0.8+)

\newpage
\appendix

\section{Computational Environment}

System Specifications:
\begin{itemize}
    \item Python version: 3.13.x
    \item Operating System: Linux (Ubuntu 24.04 LTS)
    \item RAM: 32 GB
    \item CPU: 32 cores
\end{itemize}

Reproducibility: All analyses use \texttt{random\_state=42} for reproducibility. Complete package versions available in \texttt{requirements.txt}.

\section{Code Availability}

Complete analysis code is available in \texttt{final\_project\_solution.ipynb, see view report on final project} with:
\begin{itemize}
    \item Detailed comments explaining each step
    \item Modular functions for reusability
    \item Error handling and validation checks
    \item High-resolution figure outputs (300 DPI)
    \item https://frankfurtmacmoses.github.io/bio-informatics/
\end{itemize}

\vspace{2cm}
\noindent
\textbf{Report Prepared By:} Olawole Frankfurt Ogunfunminiyi, Niraj Kc \\
\textbf{Contact:} \href{mailto:frankfurtmacmoses@gmail.com}{frankfurtmacmoses@gmail.com} \\
\textbf{Course Instructor:} Dr. Seungchan Kim \\
\textbf{Submission Date:} November 2025

\vspace{1cm}
\noindent
\end{document}
